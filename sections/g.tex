\entry{Game Mode}{
	The GameMode Class is responsible for setting 
	the rules of the game that is being played. 
	The rules can include how players join the game, 
	whether or not a game can be paused, and level 
	transitions, as well as any game-specific 
	behavior such as win conditions.
	
	You can set the default GameMode in the 
	Project Settings, but can override it on 
	a per-Level basis. Regardless of how you 
	choose to implement the GameMode, there 
	is always only one GameMode present 
	per-level. In a multiplayer game, the 
	GameMode only exists on the server and 
	the rules are replicated (sent) to each 
	of the connected clients.
}

\entry{Game State}{
	The GameState contains the information 
	that you want replicated to every client 
	in a game, or more simply it is 
	'The State of the Game' for everyone connected.

	It often contains information about 
	game score, whether a match has started 
	or not, how many AI \ref{AI} to spawn \ref{Spawn} 
	based upon the number of players in the world, 
	and other game specific information.
	
	For multiplayer games, there is one instance 
	of the GameState on each player's machine with 
	the server's instance being the authoritative 
	one (or the one that clients get their 
	updated information from).
}

\entry{Gameplay Framework}{
	Contains the information about core systems, 
	such as game rules, player input and controls, 
	cameras, and user interfaces.

	To show how the different pieces of 
	game framework interlock, here is a simple example. 
	Imagine a simple concept for a game, where a 
	rabbit races a snail. The base of the game 
	framework is the GameMode \ref{Game Mode}. 
	The GameMode sets the rules for the game, 
	like the rule that whichever player crosses 
	the finish line first is the winner. 
	It also handles spawning the players.

	A player is set up in a \texttt{PlayerController} \ref{Controller}, 
	which can possess a Pawn \ref{Pawn}. The Pawn is the 
	physical representation of a player in the game, 
	while the Controller possesses the Pawn and can set 
	rules for its behavior. In our example, there would 
	be two Pawns, one for the snail and one for the rabbit. 
	The rabbit would actually be a Character \ref{Character}, 
	a special subclass of Pawn which has built-in movement 
	functionality including running and jumping. 
	The snail, on the other hand, has a different style of 
	movement, so it could extend directly from the Pawn class.

	A Pawn can contain its own rules for movement 
	and other game logic, but that functionality can 
	also be included in a Controller. A Controller 
	can either be a \texttt{PlayerController} taking input from a 
	human player or an \texttt{AIController} with automated control 
	by the computer. In this example, the player would be 
	controlling the snail, so the snail Pawn would be possessed 
	by a \texttt{PlayerController}. The rabbit would be controlled 
	by AI \ref{AI}, with rules for when it should stop, sprint, or 
	nap all being set up in an \texttt{AIController} which possesses 
	the rabbit Character. Only human players care about the 
	view provided by a Camera \ref{Camera}, so only one \texttt{CameraComponent} 
	in the snail Pawn would be used by the \texttt{PlayerCamera}.

	During gameplay, input from the player would move the snail around the 
	map, while the HUD \ref{HUD} was overlaid on the 
	view provided by the Camera, showing information 
	about who was currently in first place, and the 
	race time that had elapsed.
}

\entry{Generator}{
	Within the Environmental Query System (EQS), 
	Generators are used to produce the locations 
	or Actors \ref{Actor} (referred to as Items) 
	that will be tested and weighted, with the 
	highest weight location or Actor being returned 
	to the Behavior Tree. There are different types 
	of Generators that you can use to retrieve 
	information and they can be created either 
	in Blueprint or C++. 
}

\entry{Gizmo}{
	3D gizmos help you move, rotate, or scale a set of objects along a 3D axis or plane.
	\fig{https://help.autodesk.com/cloudhelp/2020/ENU/AutoCAD-Core/images/GUID-A67A1239-F373-43DF-A434-0763DC2930AE.png}{Autodesk}{gizmo-autodesk.png}
}