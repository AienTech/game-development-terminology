\entry{Point light}{
	Point Lights work much like a real-world 
	light bulb, emitting light in all directions 
	from the light bulb's tungsten filament. 
	However, for the sake of performance, 
	Point Lights are simplified down emitting light 
	equally in all directions from just a single point in space.
}

\entry{Polish Pass}{
	Additional effects, audio and volumes are added, 
	and final assets and details are tweaked.
}

\entry{Polygon}{
	A polygon is a coplanar set of faces \ref{Face}. 
	In systems that support multi-sided faces, polygons and faces 
	are equivalent. However, most rendering hardware supports only 
	3- or 4-sided faces, so polygons are represented as multiple 
	faces. Mathematically a polygonal mesh may be considered 
	an unstructured grid, or undirected graph, with additional 
	properties of geometry, shape and topology.
}

\entry{Prototype}{
	A first or preliminary version of a device or something 
	from which other forms are developed.
}

\entry{Prototype Pass}{
	The first pass \ref{Workflow Pass} when designing 
	a level \ref{Level} is the prototype \ref{Prototype} pass. 
	In this pass, a very general level \ref{Level} prototype is laid out 
	using either Brushes or basic geometric Static Meshes 
	such as cubes and spheres. Basic materials \ref{Material} can be 
	applied to any Static Meshes, but this is more for 
	being able to differentiate different objects or areas 
	within your level \ref{Level}, rather than for any aesthetic effect. 
	Basic lighting is also added to the level \ref{Level} at this point.

	The prototyping pass is meant to be a quick 
	and simple process where the basic gameplay 
	areas are roughed out. This allows for testing 
	gameplay within the level \ref{Level}, and observing things 
	like the layout and relative sizing of areas 
	within the level \ref{Level}, as well as how that affects 
	players' movement throughout the game. By the 
	end of this pass, you can have a good sense of 
	the playability and environment of your area.
}