\entry{Material}{
	A Material is an asset that can be 
	applied to a mesh \ref{Mesh} to control the visual 
	look of the scene. At a high level, it is probably 
	easiest to think of a Material as the "paint" 
	that is applied to an object.
}

\entry{Mesh}{
	In 3D computer graphics and solid modeling, a polygon mesh is 
	a collection of vertices, edges and faces that defines the shape 
	of a polyhedral object. The faces usually consist of triangles 
	(triangle mesh), quadrilaterals (quads), or other simple convex 
	polygons (n-gons), since this simplifies rendering, but may also 
	be more generally composed of concave polygons, or even polygons 
	with holes.
	\fig{https://upload.wikimedia.org/wikipedia/commons/thumb/f/fb/Dolphin\_triangle\_mesh.png/250px-Dolphin\_triangle\_mesh.png}{Wikipedia}{250px-Dolphin_triangle_mesh.png}
	A mesh is consisted of 5 components:
	\fig{https://upload.wikimedia.org/wikipedia/commons/thumb/6/6d/Mesh\_overview.svg/720px-Mesh\_overview.png}{Wikipedia}{720px-Mesh_overview.png}
	\begin{itemize}
		\item Vertices \ref{Vertex}
		\item Edges \ref{Edge}
		\item Faces \ref{Face}
		\item polygons \ref{Polygon}
		\item Surfaces \ref{Surface}
		\item Materials \ref{Material}
		\item UV Coordinates \ref{UV Coordinates}
	\end{itemize}
}

\entry{Meshing Pass}{
	Replacing the basic meshes \ref{Mesh} or primitives from 
	the prototype phase \ref{Prototype Pass} with near final 
	asset and applying basic materials \ref{Material}.
}