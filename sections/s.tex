\entry{Sky light}{
	The Sky Light captures the distant parts of 
	your level \ref{Level} and applies that to the scene as 
	a light. That means the sky's appearance and 
	its lighting/reflections will match, even if 
	your sky is coming from atmosphere, or layered 
	clouds on top of a skybox, or distant mountains. 
	You can also manually specify a cubemap to use.
}

\entry{Spot light}{
	A Spot Light emits light from a single point 
	in a cone shape. Users are given two cones 
	to shape the light - the Inner Cone Angle and Outer Cone Angle. 
	Within the Inner Cone Angle, the light achieves full 
	brightness. As you go from the extent of the inner 
	radius to the extents of the Outer Cone Angle, 
	a falloff takes place, creating a penumbra, or softening 
	around the Spot Light's disc of illumination. 
	The Radius of the light defines the length of the cones. 
	More simply, this will work like a flash light or 
	stage can light.
}

\entry{Surface}{
	More often called smoothing groups, are useful, but not required 
	to group smooth regions. Consider a cylinder with caps, 
	such as a soda can. For smooth shading of the sides, 
	all surface normals must point horizontally away from the center, 
	while the normals of the caps must point straight up and down. 
	Rendered as a single, Phong-shaded surface, the crease vertices \ref{Vertex} 
	would have incorrect normals. Thus, some way of determining where 
	to cease smoothing is needed to group smooth parts of a mesh \ref{Mesh}, 
	just as polygons \ref{Polygon} group 3-sided faces \ref{Face}. As an alternative 
	to providing surfaces/smoothing groups, a mesh may contain other 
	data for calculating the same data, such as a splitting angle 
	(polygons with normals above this threshold are either 
	automatically treated as separate smoothing groups or some 
	technique such as splitting or chamfering is automatically 
	applied to the edge between them). Additionally, 
	very high-resolution meshes are less subject to issues that would 
	require smoothing groups, as their polygons \ref{Polygon} are so 
	small as to make the need irrelevant. Further, another alternative 
	exists in the possibility of simply detaching the surfaces 
	themselves from the rest of the mesh. Renderers do not attempt to 
	smooth edges across noncontiguous polygons.
}