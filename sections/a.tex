\entry{Action Mapping}{
	Action mappings are inputs that only output 
	execution pins. Triggering these events can 
	run new lines of blueprint code. In contrast
	to Axis Mappings \ref{Axis Mapping} they are
	executed only once and thus are simpler
	in the context of the key definition.

	Actions that are bound only to a pressed 
	or released event will fire every time any 
	key that is mapped to it is pressed/released. 
	However, in the case of Paired Actions 
	(actions that have both a pressed and a released 
	function bound to them) we consider the first key 
	to be pressed to have captured the action. 
	Once a key has captured the action the other 
	bound keys’ press and release events will be 
	ignored until the capturing key has been released.
}

\entry{Actor}{
	An Actor is any object that can be 
	placed into a level \ref{Level}. Actors are 
	a generic Class that support 3D transformations such as 
	translation, rotation, and scale. 
	Actors can be created (spawned) and destroyed 
	through gameplay code (C++ or Blueprints \ref{Blueprint}). 
	In C++, AActor is the base class of all Actors.
}

\entry{AI}{
	Artificial intelligence (AI), 
	is intelligence demonstrated by machines, 
	unlike the natural intelligence displayed 
	by humans and animals.

	Creating Artificial Intelligence (AI) 
	for characters \ref{Character} or other entities in your 
	projects \ref{Project} in Unreal Engine 4 
	(UE4) is accomplished through multiple 
	systems working together. From a Behavior Tree  \ref{Behavior Tree}
	that is branching between different decisions or 
	actions, running a query to get information about 
	the environment through the 
	Environment Query System (EQS) \ref{Environment Query System (EQS)}, 
	to using the AI Perception system \ref{AI Perception System} to retrieve sensory 
	information such as sight, sound, or damage information. 
	all of these systems play a key role in creating believable AI 
	in your projects. Additionally, all of these tools 
	can be debugged with the AI Debugging tools, 
	giving you insight into what the AI is 
	thinking or doing at any given moment. 
}

\entry{AI Perception System}{
	In addition to Behavior Trees \ref{Behavior Tree}
	which can be used to make decisions on which logic 
	to execute, and the 
	Environmental Query System (EQS) \ref{Environment Query System (EQS)} 
	used to retrieve information about the environment; 
	another tool you can use within the AI framework which 
	provides sensory data for an AI \ref{AI} is the AI Perception System. 
	This provides a way for Pawns \ref{Pawn} to receive data 
	from the environment, such as where 
	noises are coming from, if the AI was 
	damaged by something, or if the AI sees 
	something. This is accomplished with the 
	AI Perception Component that acts as a 
	stimuli listener and gathers registered Stimuli Sources.
}

\entry{Axis Mapping}{
	In general, Axis Mappings allow us to map keyboard, 
	mouse, and controller inputs to a "friendly name" 
	that can later be bound to game behavior, such as movement. 
	Axis Mappings are continuously polled, allowing for seamless 
	movement transitions and smooth game behavior. 
	Hardware axes (such as controller joysticks) provide 
	degrees of input, rather than discrete input 
	(1 - pressed vs. 0 - not pressed). While controller joystick 
	input methods are effective at providing scalable movement 
	input, Axis Mappings can also map common movement keys, 
	like WASD to continuously-polled game behavior.
}